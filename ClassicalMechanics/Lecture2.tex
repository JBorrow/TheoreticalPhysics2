\section{Generalised co-ordinates}

We always restrict the number of variables to a minimum because this means that
we have fewer equations to solve (and mistakes to make). The minimum number of
co-ordinates is equal to the number of degrees of freeom.

We regard co-ordinates and generalised velocities as independent. If we think
simultaneously, then there is no reason that $\dot{x}$ is related to $x$. We
have the relation:
$$
	\bm{v}_i = \dot{\bm{r}}_i = \Sigma_k \frac{\partial \bm{r}_i}{\partial q_k}
	\dot{q_k} + \frac{\partial r_i}{\partial t}
$$
Which we can reduce to the `cancelling the dots' formula:
$$
	\frac{\partial \dot{\bm{r}_i}}{\partial \dot{q_k}} =
	\frac{\partial \bm{r}_i}{\partial q_k}
$$
